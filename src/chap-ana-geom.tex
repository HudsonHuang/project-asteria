\chapter{解析几何}
\section{平面上直线的方程;无穷远处的东西}
(接下来好像是高考能用得到的黑科技,但是需要先看前面的线性代数,以及理论力学关于自由度的内容)

在二维平面上,一个点有两个自由度,也就是说确定它的坐标需要两个参数。无论用直角坐标、极坐标还是其他奇怪的坐标,要确定一个点在平面上的位置都需要两个参数,而且这两个参数可以(一一对应地)表示平面上的所有点。(极坐标规定$r \ge 0$,$\theta \in [0,2 \pi)$,且在$r=0$时$\theta=0$,否则就不是一一对应。其他奇怪的坐标需要更奇怪的定义域)

平面上的一条直线在直角坐标系中可以表示为$A x+B y+C=0$,$A,B$不全为$0$。这是一个约束条件,剩下一个自由度,而确定一个点在直线上的位置确实需要一个参数。比如确定了$x$,用这个方程可以解出$y$,那么点在平面上的位置就确定了。

为什么说这个方程表示一条直线呢?这就不是数学的问题了,我们可以随便定义“直线”这个概念。方程$A x+B y+C=0$放到直角坐标系里看起来是直的,符合我们的直觉。以后我们还会遇到把$(x-x_0)^2+(y-y_0)^2=r^2$这样的方程称为“直线”的情况。

可以把$A x+B y+C=0$改写为$\xi x+\eta y=1$,其中$\xi=-\frac{A}{C}$,$\eta=-\frac{B}{C}$,而且$\xi$和$\eta$不全为$0$。每组$(\xi,\eta)$对应一条不过原点的直线,也就是说平面上所有不过原点的直线可以用两个参数表示。上面的$A,B,C$三个参数中,只有两个是独立的。

($\xi$读作xi,克赛,$\eta$读作eta,伊塔,它们分别与英文字母x、y对应)

但是“不全为$0$”和“不过原点”放在这里很不爽!先考虑这样的情况:$\xi=\eta=a>0$,它表示一条斜着的直线。$a$越小,这条直线离原点越远。如果$a=0$,可以认为它表示一条无穷远处的直线。无穷远处的直线只有一条,不管它是从哪个方向移到无穷远的,都是同一条。

如果$\xi=\infty$,$\eta=\infty$,那么它表示一条过原点的直线。但是过原点的直线有无穷多条,好在无穷大也是有大小之分的,这条直线的斜率$k=-\frac{\xi}{\eta}$,跟普通直线的斜率定义一样。(虽然我还没讲过无穷大的严格定义是什么,两个无穷大的比值应该在讲极限的时候讲。$k$可能是$0$或者$\infty$)

平面上的一个点可以表示为(x,y),跟上面对应,$x=y=0$时,它就是原点,只有一个;$x=\infty$,$y=\infty$时,它表示一个无穷远处的点,或者说无穷远直线上的点,这样的点有无穷多个。

这样就有了奇怪的对称性:过每个点的直线都有无穷多条,每条直线上的点都有无穷多个,过两个点的直线有且只有一条,两条直线的交点有且只有一个。(包括原点、无穷远点、过原点的直线、无穷远直线)

这些无穷远处的东西是射影几何研究的内容,我们以后再讲,一般的解析几何不会考虑。高考很喜欢在这种地方挖坑。

顺便说一句,一个复数相当于两个实数,所以用一个复参数可以表示复平面上的所有点。
\section{空间中平面和直线的方程}
在三维空间中,一个点有三个自由度,可以类似地发现方程$A x+B y+C z+D=0$表示一个平面,它是一个约束条件,剩下两个自由度。三维空间中所有不过原点的平面可以用三个参数表示。

它的法向量$\mathbf{n}$平行于$(A,B,C)$。如果有一个矢量$\mathbf{a}$满足$\mathbf{n} \cdot \mathbf{a}=A a_x+B a_y+C a_z=0$,那么$\mathbf{a}$与平面平行。

如果一个不过原点的平面在$x,y,z$轴上的截距分别是$x_0,y_0,z_0$,它的方程就是$\frac{x}{x_0}+\frac{y}{y_0}+\frac{z}{z_0}=1$,但是过原点的平面就不能这么算了。(化竞似乎学过一种叫晶格指数的东西)

平面上点到直线的距离是$\frac{|A x_0+B y_0+C|}{\sqrt{A^2+B^2}}$,平行直线的距离是$\frac{|C_1-C_2|}{\sqrt{A^2+B^2}}$。类似地,空间中点到平面的距离是$\frac{|A x_0+B y_0+C z_0+D|}{\sqrt{A^2+B^2+C^2}}$,平行平面的距离是$\frac{|D_1-D_2|}{\sqrt{A^2+B^2+C^2}}$。这些公式很容易推广到高维空间。

三维空间中的直线需要两个约束条件,比如
\begin{equation*}
\begin{cases}
A_1 x+B_1 y+C_1 z+D_1=0 \\
A_2 x+B_2 y+C_2 z+D_2=0
\end{cases}
\end{equation*}

也就是两个平面相交得到一条直线。但是如果这两个平面平行,就不能表示一条直线了。(不考虑无穷远处的东西)

也可以用矢量表示:$\mathbf{a}+\lambda \mathbf{b}$,用分量表示就是
\begin{equation*}
\begin{cases}
x=a_x+\lambda b_x \\
y=a_y+\lambda b_y \\
z=a_z+\lambda b_z
\end{cases}
\end{equation*}

这样用了六个参数$a_x,a_y,a_z,b_x,b_y,b_z$来确定一条直线,而$\lambda$用来确定直线上的点。但是把$\mathbf{b}$乘上一个倍数,或者把$\mathbf{a}$换成$\mathbf{a}+\mathbf{b}$,直线不会改变,这说明有两个参数不是独立的。

在三维空间中表示一条直线只需要四个独立的参数,可以这样想:两点确定一条直线,这两个点共有六个自由度。但是把第一个点沿着直线移动,直线不会改变,第二个点也是一样,这样就少了两个自由度,剩下四个自由度。

空间中点到直线的距离好像没有容易背的公式,可以用投影来算。
\section{三点共线和相关的东西}
在平面上,三个点的坐标为$P_1(x_1,y_1),P_2(x_2,y_2),P_3(x_3,y_3)$,如果它们共线,可以表示为$\frac{y_1-y_2}{x_1-x_2}=\frac{y_1-y_3}{x_1-x_3}$。但是这样对下标$1,2,3$不对称,还会遇到分母为$0$的问题。一种更炫酷的表示方法是
\begin{equation*}
\begin{vmatrix}
x_1 & y_1 & 1 \\
x_2 & y_2 & 1 \\
x_3 & y_3 & 1
\end{vmatrix}=0
\end{equation*}

如果你熟悉行列式的运算技巧,可以理解为第三行等于$\lambda$倍的第一行加$1-\lambda$倍的第二行,也就是说$P_3$在直线$P_1 P_2$上。

现在我们从第一种表示推导第二种表示。设$\frac{y_1-y_2}{x_1-x_2}=\frac{y_1-y_3}{x_1-x_3}=k$,那么$k x_1-y_1=k x_2-y_2$。设$k x_1-y_1=m$,那么$x_1 \cdot \frac{k}{m}+y_1 \cdot -\frac{1}{m}+1 \cdot -1=0$。同理,$x_2 \cdot \frac{k}{m}+y_2 \cdot -\frac{1}{m}+1 \cdot -1=0$,$x_3 \cdot \frac{k}{m}+y_3 \cdot -\frac{1}{m}+1 \cdot -1=0$。

也就是说,存在一组实数$(a_1,a_2,a_3)=(\frac{k}{m},-\frac{1}{m},-1)$,它们满足线性齐次方程组
\begin{equation*}
\begin{cases}
x_1 a_1+y_1 a_2+a_3=0 \\
x_2 a_1+y_2 a_2+a_3=0 \\
x_3 a_1+y_3 a_2+a_3=0
\end{cases}
\end{equation*}

注意在这个方程组中,$x$和$y$是已知的系数,$a$是未知量。在线性代数那章讲过,要让这个方程组有解,系数行列式必须为$0$,于是就有了三点共线的第二种表示。

举个栗子:已知直线$l$经过$A(1,2),B(3,4)$,它与直线$y=5x+6$交于$C$,求$C$的坐标。(我编出来的系数就是这么普通)

我们不用算出$l$的方程,直接用行列式来做。设$C(x,5x+6)$,$\begin{vmatrix}
1 & 2 & 1 \\
3 & 4 & 1 \\
x & 5x+6 & 1
\end{vmatrix}=0$,把行列式算出来得到$8x+10=0$,$x=-\frac{5}{4}$,$C(-\frac{5}{4},-\frac{1}{4})$。

平面上的三个点共有六个自由度,三点共线是一个约束条件,剩下五个自由度。我已经在很多地方强调过,自由度的数量可以用来判断一道题能不能做出来,一般情况下$n$个方程能且只能解出$n$个未知数。高中物理题有时候会出现多出来的条件,有时候会自相矛盾,更麻烦的是有时候条件不够,而且真正的高考和竞赛里确实出现过这样的情况,我也不知道该怎么办。

如果把刚才的方程组看成关于$k$和$m$的方程组,那么有两个未知数,但是有三个方程,一般是无解的,除非出现特殊情况,这里的特殊情况就是三点共线。

如果有三条直线$l_1:\xi_1 x+\eta_1 y=1,l_2:\xi_2 x+\eta_2 y=1,l_3:\xi_3 x+\eta_3 y=1$,这三条线共点可以类似地表示为
\begin{equation*}
\begin{vmatrix}
\xi_1 & \eta_1 & 1 \\
\xi_2 & \eta_2 & 1 \\
\xi_3 & \eta_3 & 1
\end{vmatrix}=0
\end{equation*}

在三维空间中,四点共面可以类似地写成
\begin{equation*}
\begin{vmatrix}
x_1 & y_1 & z_1 & 1 \\
x_2 & y_2 & z_2 & 1 \\
x_3 & y_3 & z_3 & 1 \\
x_4 & y_4 & z_4 & 1
\end{vmatrix}=0
\end{equation*}

但是如果不熟悉四阶行列式的计算,可能还是先用三个点求出平面的方程,再把第四个点代进去比较方便。你猜四个平面共点怎么表示?

三维空间中的三点共线和三面共线都相当于两个约束条件,好像没有比较对称的表示方法,如果有人知道欢迎来告诉我。
\section{叉积}
二维矢量的叉积是一个标量,$\mathbf{a} \times \mathbf{b}=a_x b_y-a_y b_x$。它满足反交换律$\mathbf{a} \times \mathbf{b}=-\mathbf{b} \times \mathbf{a}$和分配律$\mathbf{a} \times (\mathbf{b}+\mathbf{c})=\mathbf{a} \times \mathbf{b}+\mathbf{a} \times \mathbf{c}$。

三维矢量的叉积是一个矢量,$\mathbf{a} \times \mathbf{b}=(a_y b_z-a_z b_y,a_z b_x-a_x b_z,a_x b_y-a_y b_x)$。它满足反交换律和分配律,一般不满足结合律。用多了之后这个公式就背下来了。

设$\mathbf{c}=\mathbf{a} \times \mathbf{b}$,你可以验证一下$\mathbf{c} \perp \mathbf{a}$且$\mathbf{c} \perp \mathbf{b}$。在右手系中,$\mathbf{c}$的方向可以用右手螺旋来判断:把右手四指指向$\mathbf{a}$的方向,然后转向$\mathbf{b}$的方向,大拇指的方向就是$\mathbf{c}$的方向。

很多物理公式要用叉积才能严格表示出矢量的方向,比如洛伦兹力$\mathbf{F}=q \mathbf{v} \times \mathbf{B}$,安培力$\mathbf{F}=I \mathbf{L} \times \mathbf{B}$(电流$I$是标量,而$\mathbf{L}$的方向是电流方向),力矩$\mathbf{M}=\mathbf{r} \times \mathbf{F}$等等。但是如果你习惯用右手定则左手定则之类的判断它们的方向,那就不用管右手螺旋了,以免搞混。

如果平行四边形的两条邻边分别是矢量$\mathbf{a},\mathbf{b}$,那么它的面积是$|\mathbf{a} \times \mathbf{b}|$(二维情况下是取绝对值,三维情况下是取模长),三角形的面积要除以$2$。

如果平行六面体从一个顶点出发的三条边分别是矢量$\mathbf{a},\mathbf{b},\mathbf{c}$,那么它的体积是$|\mathbf{a} \times \mathbf{b} \cdot \mathbf{c}|=\left| \begin{vmatrix}
a_x & a_y & a_z \\
b_x & b_y & b_z \\
c_x & c_y & c_z
\end{vmatrix} \right|$,三棱柱的体积要除以$2$,三棱锥的体积要除以$6$。

$\mathbf{a} \times \mathbf{b} \cdot \mathbf{c}$称为矢量三重积,它是一个标量,可正可负。你可以验证一下矢量三重积和行列式的结果是一样的,并且$\mathbf{a} \times \mathbf{b} \cdot \mathbf{c}=\mathbf{b} \times \mathbf{c} \cdot \mathbf{a}=-\mathbf{b} \times \mathbf{a} \cdot \mathbf{c}$。

高考立体几何经常让你算二面角,我个人用的都是建系暴算。知道一个面的两条边对应的矢量,就能用叉积算出它的法向量,而不用解方程组。如果觉得直接算出来的法向量太难看,可以乘一个(正的)倍数变得好看一点。而且这样算出来的法向量的方向可以用右手螺旋确定,就不用凭感觉判断是锐角还是钝角了。算出法向量之后记得在图中对照一下,特别是平行于坐标轴或者垂直于坐标轴的情况。如果你不熟悉这种做法,可以随便找几道立体几何题试一下。

一条直线把平面分成两部分,设直线上有一点$O$,平面上有一点$P$,可以用叉积表示$P$在直线的哪一侧:取平行于直线的向量$\mathbf{n}$,然后计算$\Delta=\mathbf{OP} \times \mathbf{n}$,如果$\Delta>0$,那么$P$在直线的一侧;如果$\Delta<0$,那么$P$在另一侧;如果$\Delta=0$,那么$P$在直线上。

而左侧和右侧是人为定义的,如果定义$\Delta>0$的部分是左侧,那么$\Delta<0$就是右侧。

一个平面把三维空间分成两部分,可以在平面上取两个方向不同的向量$\mathbf{n}_1,\mathbf{n}_2$,计算三重积$\Delta=\mathbf{OP} \times \mathbf{n}_1 \cdot \mathbf{n}_2$,然后进行类似的分类。在3D的游戏中,我们经常看到一个面的两侧显示不同的颜色,电脑就是这样判断一个面的两侧的。
\section{坐标变换的代数性质}
平移、旋转不会改变线段的长度,也不会改变两条直线的夹角。如果把坐标轴缩放一下,线段的长度和两条直线的夹角会改变,但是原来平行的线段仍然平行。

平移、旋转和缩放统称为线性变换,或者叫仿射变换。在线性代数那章讲过,坐标变换对矢量的效果可以用矩阵表示,但是平移对矢量是没有用的,所以只有旋转和缩放。现在我们考虑点而不是矢量,平移会改变点相对于坐标原点的位置。平面上的线性变换可以写成
\begin{equation*}
\begin{cases}
x'=a_{1 1} x+a_{1 2} y+c_1 \\
y'=a_{2 1} x+a_{2 2} y+c_2
\end{cases}
\end{equation*}

$c_1$和$c_2$表示平移的效果。

线性变换不会改变曲线的方程的次数,直线变换之后还是直线,二次曲线变换之后还是二次曲线。事实上,椭圆(包括圆)变换之后还是椭圆,抛物线变换之后还是抛物线,双曲线变换之后还是双曲线。因此线性变换把所有二次曲线分成了椭圆、抛物线、双曲线三类。(除了一些特殊情况,或者说\emph{退化}的情况,包括空集、一个点、一条直线、两条平行直线、两条相交直线)

如果二次曲线的方程是$A x^2+B x y+C y^2+D x+E y+F=0$,设$\Delta=\begin{vmatrix}
2 A & D & B \\
D & 2 C & E \\
B & E & 2 F
\end{vmatrix}$,$\delta=\begin{vmatrix}
2 A & B \\
B & 2 C
\end{vmatrix}$,可以这样判断:当$\Delta \neq 0$时,如果$\delta>0$,则是椭圆或者空集(具体是哪种可以配方试试看);如果$\delta=0$,则是抛物线;如果$\delta<0$,则是双曲线;当$\Delta=0$时,则是退化的情况。椭圆、抛物线、双曲线分别对应$\delta>0$、$\delta=0$、$\delta<0$三种情况,所以二次曲线只有这三类。

线性变换不会改变两条曲线的交点数量,特别是相切的曲线变换之后还是相切。与椭圆有关的问题可以缩放坐标轴,把它变成一个圆,特别是与平行和相切有关的问题。

有人认为,几何学研究的是变换下的不变性,不同的几何学研究不同的变换。比如欧氏几何研究长度、角度这些平移、旋转(或者叫正交变换)下的不变性,射影几何研究平行、交比这些射影变换下的不变性,拓扑几何研究一个面上有几个洞这些拓扑变换下的不变性。而每种变换都是可以用代数方法描述的,这样就在代数与几何之间建立起了联系。反过来也可以用几何方法来研究代数问题,比如一些不满足交换律的代数可以用空间中的旋转(或者爸爸和妈妈)表示,对$c$取模的代数可以用周长为$c$的圆圈表示。这就是传说中的\emph{爱尔兰根纲领},是现代数学研究十分重要的方向。
\section{五个点确定一条二次曲线}
刚才讲了平面上二次曲线一般的方程:$A x^2+B x y+C y^2+D x+E y+F=0$。如果$A,B,\dots,F$同时乘一个倍数,表示的还是原来的二次曲线,所以这六个参数中只有五个是独立的。因此我们可以猜:五个点确定一条过这五个点的二次曲线,道理跟两个点确定一条直线差不多。

我们在平面上取五个点$P_1,P_2,\dots,P_5$,要求任意三个点不共线。过$P_1,P_2$作直线$l_1$,设它的方程为$l_{1 1} x+l_{1 2} y+l_{1 3}=0$。我们把代数式$l_{1 1} x+l_{1 2} y+l_{1 3}$也称为$l_1$,也就是说直线$l_1$的方程为$l_1=0$。(接下来不区分直线$l_1$,代数式$l_1$和方程$l_1=0$,可以根据上下文判断)然后过$P_3,P_4$作$l_2$,过$P_1,P_3$作$l_3$,过$P_2,P_4$作$l_4$。

现在看这个式子:$S=l_1 l_2+\lambda l_3 l_4$,$\lambda$是一个待定系数。这是关于$x,y$的二次式(除了退化的情况),而且把$P_1,P_2,P_3,P_4$代进去都得到$0$,也就是说这是一条过$P_1,P_2,P_3,P_4$的二次曲线。既然任意三个点不共线,$P_5$代进去一般不能得到$S=0$,这时令$S=0$可以解出$\lambda$,也就确定了这条二次曲线。所以说,如果五个点中任意三个点不共线,那么有且只有一条二次曲线过这五个点。

用类似的方法可以证明,如果九个点中任意三个点不共直线(或者说一次曲线),并且任意六个点不共二次曲线,那么有且只有一条三次曲线过这九个点。你可以想想更高次和零次的情况。

在高斯的时代,九个点能否确定一条三次曲线引起过很大的争论,况且实际问题中的九个点一般会有六个点共二次曲线。这个问题最简单的想法是把九个点代进三次曲线的一般方程,然后用九个方程解出九个待定系数。然而这九个方程不一定是独立的,这一点用肉眼当然很难看出来,因此这个问题推动了线性代数的发展,这又是一个代数和几何相结合的例子。

(以后应该专门有一章射影几何,奇怪的空间曲面也要以后再讲)

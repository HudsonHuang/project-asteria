\chapter{推荐书目}
\nocite{lz2004gdsx}
\nocite{riley2006mathematical}
\nocite{wcs1999sxwlff}
\nocite{zlq2006sxfx}
\nocite{wl2014sfxzdfl}
\nocite{gelbaum2003counterexamples}
\nocite{nahin2014inside}
\nocite{mahajan2010street}
\nocite{qws2010gdds}
\nocite{von2013modern}
\nocite{krauth2006statistical}
\nocite{ld2007lx}
\nocite{shc2006jdlx}
\nocite{gsh2008ddlx}
\nocite{ld2012cl}
\nocite{cbq2001dcxztyj}
\nocite{wzc2008rlxtjwl}
\nocite{shc2011tjlx}
\nocite{griffiths2004introduction}
\nocite{sakurai2011modern}
\nocite{zkh1991dxhbdlwlx}
\nocite{zwiebach2004first}
\nocite{horowitz1980art}
\nocite{lyq2001jglxjc}
\renewcommand{\bibname}{}
\patchcmd{\thebibliography}{\chapter*}{}{}{}
\bibliographystyle{unsrt}
\bibliography{bib}

上面这些书可以让大家感受一下大学里的数学和物理是什么样的,虽然竞赛、自招和高考不太可能直接考这些东西,但是看一看这些东西可以帮助我们梳理高中知识的体系,并且了解高中知识的局限性。大学的教材有很多版本,但总是大同小异的。
